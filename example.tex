\documentclass[12pt,oneside]{fithesis2}
\usepackage[english]{babel}
\usepackage[utf8]{inputenc}
\usepackage[T1]{fontenc}
\usepackage[
  scaled=0.86
]{berasans}
\usepackage[
  scaled=1.03
]{inconsolata}
\usepackage[
  plainpages = false,
  pdfpagelabels
]{hyperref}
\usepackage{blindtext}

\thesislang{en}
\thesistitle{Sample Thesis}
\thesissubtitle{Diploma Thesis}
\thesisstudent{Bc. Martin Jordán}
\thesiswoman{true}
\thesisfaculty{fi}
\thesisyear{Spring \the\year}
\thesisadvisor{Ing. Leonard Walletzký, PhD.}

\begin{document}
  \FrontMatter
    \ThesisTitlePage
    \begin{ThesisDeclaration}
      \DeclarationText
      \AdvisorName
    \end{ThesisDeclaration}
    \begin{ThesisThanks}
      I would like to thank my supervisor Ing. Leonard Walletzký, PhD. \,\dots
    \end{ThesisThanks}
    \begin{ThesisAbstract}
      This thesis deals with analysing, modelling and automating specific processes in the financial department. The automation will be carried out by a web application that is intended to be used by a very limited number of financial department employees. The first part introduces the company as well as the reasons why this application is being implemented. The rest of the thesis describes the process of designing, implementing and deploying the application.
    \end{ThesisAbstract}
    \begin{ThesisKeyWords}
      24i, Automation, Finance, Python, JavaScript, React, GatsbyJS, Serverless, REST API
    \end{ThesisKeyWords}
    \tableofcontents
%   \listoftables
%   \listoffigures
  
  \MainMatter
    \chapter{Introduction}
    When assessing whether a company is successful or not, we can look at many different metrics ranging from market share, through the number of products sold over a specific period of time all the way to customer satisfaction. However, there is one metric that stands above all  - financial success. After all, the end goal of every company is to generate, continually increase and maximize profit over a sustained period of time.
    \par
    In the early stages of a company, having positive financial results can be difficult to achieve. Sometimes, it is not even seen as the top priority. For startups, growth velocity and market penetration is most important. It helps them increase market share first so that they can build customer base and then use that momentum to maximize profits over the long term.
    \par
    Still, for a company to be able to expand rapidly, funds are essential. However, in such an early and growth focused stages, money can be hard to come by even in the years of highest abundance. To accelerate growth, companies are constantly searching for ways to acquire cheap capital.
    \par
    To such companies, European Union is offering funding grants in various areas of expertise. To be able to acquire these grants, smaller businesses must comply to a strict set of rules set by the European Commission.
    \par
    Alternatively, small companies turn to bigger players within the same space to sell them a stake in the company for a financial injection. This can create many different challenges, such as an incompatibility of certain systems the companies are using.
    \par
    This thesis aims to solve the two problems described above. First, it is to help the company improve it's financial state by providing insights into employee retention and the effects this has on eligibility for the funding grants. Second, it is to provide means of reducing workload on accountants by giving them a simple tool they can use to extract financial data from an accounting system that does not support desired tax form format.
    \newpage
    The content of this thesis is divided into four chapters. The first part of the thesis is dedicated to briefly introducing the company and the funding grants in general, as well as explaining why the company is applying for said grants. Furthermore, information regarding generating the tax forms is provided here. Following that, the processes that are needed to be completed in order to acquire such grant are modeled. The next chapter provides a description of the solution as well as details about managing the project. Finally, the last chapter contains the implementation details.
    \chapter{Another chapter}
    \Blindtext
    \appendix
    \chapter{First appendix}
    \Blindtext
    \chapter{Another appendix}
    \Blindtext
    % Bibliography goes here
    % Index goes here (optional)
\end{document}
